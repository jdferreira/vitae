\section{Pedagogical performance}

\subsection{Teaching}

\subsubsection{Courses}
\vspace{-1.5\baselineskip}
\begin{table}[H]
\centering
\caption{This table shows my contribution to courses from the Departamento de Informática da Faculdade de Ciências da Universidade de Lisboa. \hfill\break
\textbf{Legend}:
\textit{PL}: Prática de Laboratório (Practice lesson);
\textit{TP}: Teórico-Prática (Theoretical \& Practical lesson)}
\label{tab:courses}
\begin{tabular}{llll}
\toprule
\bfseries Course & \bfseries Position & \bfseries 2015--2016 & \bfseries 2016--2017 \\
\midrule
ASW     & PAC & \bfseries PL & \bfseries PL \\
IC      & PAC & \bfseries TP \& PL & -- \\
ITW     & PAC & \bfseries TP \& PL & \bfseries TP \& PL \\
PD      & PAC & \bfseries TP & \bfseries TP \\
Prog. I & PAC & \bfseries TP & \bfseries TP \\
\bottomrule
\end{tabular}
\end{table}

\entry
    [2015--2017]
    {ASW (Aplicações e Serviços na Web)}
\begin{itemize}
    \item \textbf{Degree}: Licenciatura em Tecnologias de Informação (2\textsuperscript{nd} year)
    \item \textbf{My lectures}: PL
    \item \textbf{Topics}:
    \begin{itemize}
        \item Web application characteristics and features;
        \item The development process of web applications;
        \item Introduction to the main server-side web technologies: resource addressing, protocols and general architecture;
        \item The various data transfer formats (XML, JSON, \emph{etc.}) and related technologies;
        \item Introduction to Web Services and Semantic Web.
    \end{itemize}
    \item \textbf{My contributions}: I produced a guide to explain HTML forms, along with JavaScript access to the content of the forms and the content of the GET arguments passed with the URL. The goal was to teach students how to deal with user-input, and how it travels through the network between web pages.
\end{itemize}


\entry
    [2015--2016]
    {IC (Interacção com Computadores)}
\begin{itemize}
    \item \textbf{Degree}: Licenciatura em Tecnologias de Informação (2\textsuperscript{nd} year)
    \item \textbf{My lectures}: TP \& PL
    \item \textbf{Topics}:
    \begin{itemize}
        \item Introduction to Human-Computer Interaction (HCI);
        \item The foundations of HCI: Human and technological aspects;
        \item The design process: user centred design, interaction design basics, guidelines for interaction design, and evaluation techniques;
        \item Models and theories: cognitive models, task analysis, dialogue notations
    \end{itemize}
    \item \textbf{My contributions}: I contributed to the class materials produced, including one of the project descriptions, which consisted in the development of an application based on web technologies that assisted people managing their kitchen (kitchen utensils, silverware, food \emph{etc}.). The goal was to teach students how to implement an application following the principles of human-computer interaction.
\end{itemize}

\entry
    [2015--2017]
    {ITW (Introdução às Tecnologias da Web)}
\begin{itemize}
    \item \textbf{Degree}: Licenciatura em Tecnologias de Informação (1\textsuperscript{st} year)
    \item \textbf{My lectures}: TP \& PL
    \item \textbf{Topics}:
    \begin{itemize}
        \item Internet and web history;
        \item Basic web concepts: the architecture, models, protocols, user agents, and transactions;
        \item Text and Hypertext Mark-up;
        \item Image Mark-up;
        \item Introduction to HTML;
        \item Introduction to Cascading Style Sheets (CSS);
        \item JavaScript concepts: flow control, data structure and objects;
        \item Processing user input in HTML Forms;
        \item Introduction to HTML5.
    \end{itemize}
    \item \textbf{My contributions}: I contributed with several HTML and CSS exercises for the students to complete in their TP lectures. The goal was to introduce the students to the development of web pages with HTML and CSS.
\end{itemize}


\entry
    [2015--2017]
    {PD (Processamento de Dados)}
\begin{itemize}
    \item \textbf{Degree}: Licenciatura em Biologia (2\textsuperscript{nd} year)
    \item \textbf{My lectures}: TP -- invited teacher
    \item \textbf{Topics}:
    \begin{itemize}
        \item Introduction to data processing;
        \item Introduction to Python (data types and data structures);
        \item Introduction to Regular Expressions;
        \item Introduction to Biomedical Web Services;
        \item Database management systems.
    \end{itemize}
    \item \textbf{My contributions}: I created all the practice class materials -- a tutorial to guide students in their project, which was to process a large collection of protein and metabolic information extracted from widely-known biomedical web services, including protein sequences and metabolic pathway data. The goal was to teach students how to process biology-related data with a programming language and an underlying database.
\end{itemize}


\entry
    [2015--2017]
    {Prog I (Programação I a outras licencituras)}
\begin{itemize}
    \item \textbf{Degree}: Several B.Sc and M.Sc offered by Faculdade de Ciências da Universidade de Lisboa
    \item \textbf{My lectures}: TP
    \item \textbf{Topics}:
    \begin{itemize}
        \item Computation: computability and Turing machines;
        \item Algorithms: exhaustive search, approximation search and bisection search;
        \item Programming methods: attribution and verification, decision, iteration and recursion, abstraction and specification, cloning;
        \item Programming languages: expressions and types, precedence and associativity, functions, scope, libraries and modules;
        \item Data structures: sequences, tuples, lists and dictionaries;
        \item Files;
        \item Software development: reading and writing, documentation, assertions and exceptions, test and debugging.
    \end{itemize}
    \item \textbf{My contributions}: I contributed to the practice class materials by suggesting new exercises and modifications to existing ones.
\end{itemize}


\subsubsection{Pedagogical surveys}
\vspace{-1.5\baselineskip}
\begin{table}[H]
\centering
\caption{Pedagogical survey results. Students could either not answer each question or answer from 1 (strong disagreement) to 4 (strong agreement). Results for each question show the average for the students that answered the question. Results from the academic year of 2016--2017 are still unavailable.\hfill\break
\textbf{Legend}:
\textit{Q1}:~Did the professor lecture with clarity?
\textit{Q2}:~Did the professor answer questions with clarity?
\textit{Q3}:~Was the professor available for outside-of-class contact \& support?
\textit{Q4}:~Was there a good pedagogical relation between professor and students?
\textit{Q5}:~What is your global appreciation of the professor?
\textit{PL}:~Prática de Laboratório (Practice lesson);
\textit{TP}:~Teórico-Prática (Theoretical \& Practical lesson);
\textit{$*$}:~Course still ongoing, results unavailable.}
\label{tab:surveys}
% \begin{tabular}{l@{\hskip4ex}ccc@{\hskip4ex}ccc}
\begin{tabular}{lcccccc}
\toprule
\bfseries 2015--2016
 % & \multicolumn{6}{c}{\bfseries 2015--2016} \\
 % & \multicolumn{3}{c}{\bfseries 1\textsuperscript{st} semester}
 % & \multicolumn{3}{c}{\bfseries 2\textsuperscript{nd} semester} \\
 & \multicolumn{2}{c}{\bfseries IC}
 & \bfseries Prog I
 & \bfseries ASW
 & \multicolumn{2}{c}{\bfseries ITW} \\
\bfseries Question
 & \bfseries TP
 & \bfseries PL
 & \bfseries TP
 & \bfseries PL
 & \bfseries TP
 & \bfseries PL \\
\midrule
Q1 & 3.86 & 3.86 & 3.69 & 3.76 & 3.68 & 3.70 \\
Q2 & 3.86 & 3.80 & 3.72 & 3.84 & 3.58 & 3.70 \\
Q3 & 3.70 & 3.71 & 3.62 & 3.87 & 3.69 & 3.94 \\
Q4 & 3.93 & 3.86 & 3.80 & 3.88 & 3.63 & 3.85 \\
Q5 & 3.87 & 3.88 & 3.79 & 3.84 & 3.70 & 3.73 \\
\bottomrule
\end{tabular}
\end{table}


\subsubsection{Teaching materials}

\entry{ASW (Aplicações e Serviços na Web)}
\begin{itemize}
    \item I produced a guide to explain HTML forms, along with JavaScript access to the content of the forms and the content of the GET arguments passed with the URL. The goal was to teach students how to deal with user-input, as well as to introduce them to the idea of information flowing with the HTTP request from a form to the next web page. This material was made available to the students through moodle.
\end{itemize}

\entry{ITW (Introdução às Tecnologias da Web)}
\begin{itemize}
    \item I contributed with several HTML and CSS exercises for the students to complete in their TP lectures. The goal was to introduce the students to the development of web pages with HTML and CSS. This material was made available to the students through moodle.
\end{itemize}

\entry{PD (Processamento de Dados)}
\begin{itemize}
    \item I created all the practice class materials -- a tutorial to guide students in their project, which was to process a large collection of protein and metabolic information extracted from widely-known biomedical web services, including protein sequences and metabolic pathway data. The goal was to teach students how to process biology-related data with a programming language and an underlying database. This material was made using google-docs and published as an HTML page, which was made available to the students through moodle.
    \item \textbf{URL}: \url{https://docs.google.com/document/d/1fhAJ3oL7Jx1W9m-zFs8ZxPpdFB59vEZHLWv9VGT0tDU/pub}
\end{itemize}


\subsection{Jury and Examinations}

\entry
    [2017]
    {M.Sc in ``Engenharia Informática e de Computadores''}
\begin{itemize}
    \item \textbf{Student}: Sebastião da Silva Freire
    \item \textbf{Title}: ``E-Sports' Ontology'
    \item \textbf{Supervisor}: Dr H Sofia Pinto (IST, Universidade de Lisboa)
    \item I was formally invited by the supervisor; there has already been a discussion on the proposal of the M.Sc, which I examined in February.
\end{itemize}

\entry
    [2017]
    {M.Sc in ``Matemática Aplicada à Economia e Gestão''}
\begin{itemize}
    \item \textbf{Student}: Catarina Nunes Valente
    \item \textbf{Title}: ``Programação em Excel para Estatística: Modelo Linear e Extensões''
    \item \textbf{Supervisor}: Dr Teresa Alpuim (FCUL, Universidade de Lisboa)
\end{itemize}

\entry
    [2016]
    {M.Sc in Bioinformatics and Computational Biology}
\begin{itemize}
    \item \textbf{Student}: Samuel Viana
    \item \textbf{Title}: ``Optimizing 16S Sequencing Analysis Pipelines''
    \item \textbf{Supervisors}: Daniel Faria (Instituto Gulbenkian de Ciência) and Catia Pesquita (FCUL, Universidade de Lisboa)
\end{itemize}


\subsection{Teaching-related activities}

\subsubsection{Invited participation in courses}

\entry
    [2015-2017]
    {Grandes Dados}
\begin{itemize}
    \item \textbf{Degree}: Doutoramento em Informática
    \item I supervised the lectures and journal club for the course regarding the topic of machine learning in Big Data.
\end{itemize}

\entry
    [2014--2015]
    {AW (Aplicações na Web)}
\begin{itemize}
    \item \textbf{Degree}: Mestrado em Engenharia Informática
    \item I presented a lecture on Semantic Web (SW) with the following topics:
    \begin{itemize}
        \item The problem of ambiguity that SW tries to solve;
        \item Rule-based inference;
        \item RDF statements;
        \item Several of the SW languages (RDF, OWL, SPARQL);
        \item Objects \emph{vs.} Classes \emph{vs.} Instances
        \item An introduction to several of the layers of the SW, including URIs, XML, RDF, Ontologies and Rules;
        \item Real-world examples of SW in action: Semantic wikis, FOAF project, RDFa, hCalendar, Linked Data Project.
    \end{itemize}
\end{itemize}

\entry
    [2013--2014]
    {Bioinformatics \& Computational Modelling}
\begin{itemize}
    \item \textbf{Degree}: PhD Program in Biological Systems -- Functional \& Integrative Genomics
    \item I presented a practical lecture on Bioinformatics, specifically on the use of Python in biomedical data processing. This included:
    \begin{itemize}
        \item Basic python datatypes and functions;
        \item Introduction to BioPython, a package with access to several functions dedicated (i)~to biological data processing and (ii)~to widely-known biomedical web services;
        \item Exercises directed at learning the inners of BioPython, specifically to process protein sequences (using web services to access the SwissProt database and the BLAST algorithm)
        \item Introduction to the Gene Ontology and Semantic Similarity
    \end{itemize}
\end{itemize}

\entry
    [2013--2014]
    {Ontologias Aplicadas às Ciências}
\begin{itemize}
    \item \textbf{Degree}: M.Sc. class available to several M.Sc. students at FCUL
    \item I collaborated on the practical classes by designing a project and supervising a group of students in implementing the project's specifications. The projects were designed so that students would develop an intuition about ontology development, ontology matching, and semantic similarity. Two of these projects resulted in publications at national and international level.
\end{itemize}

\entry
    [2011--2012]
    {Bioinformática}
\begin{itemize}
    \item \textbf{Degree}: B.Sc. class available to several B.Sc. students at FCUL
    \item I was a teaching assistant on the practical classes by invitation of the course's head professor. I created a project specification based on ontology development and text-mining, and supervised the students in their implementations.
\end{itemize}


