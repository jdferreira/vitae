\section{Education} \label{sec:education}

\entry
  [Dec. 2010 -- Jan. 2016]
  [Universidade de Lisboa]
  {Ph.D. in Computer Science – Bioinformatics}
\begin{itemize}
  \item \textbf{Thesis}: Semantic Similarity Across Biomedical Ontologies
  \item \textbf{Supervisor}: Prof. Dr Francisco M. Couto
  \item \textbf{Grade}: Aprovado com Distinção e Louvor
  \item \textbf{Abstract}: The need to compare complex entities is relevant in all the areas of science. In medicine, for example, comparing a clinical case to a database of previous cases can be extremely helpful when trying to diagnose a disease or deciding the most appropriate treatment for a patient.
  
  \hskip20pt Recent developments in knowledge representation, in particular the creation of the Web Ontology Language (OWL), have lead to a rise in the amount of knowledge that is being stored in \emph{ontologies}, which represent, in machine-readable format, the known facts about reality. With the help of ontologies, statements like ``\textbf{Influenza} is an \textbf{Infectious disease}'' can be processed by computers, which, in turn, can be used to create new knowledge. In particular, \emph{semantic similarity} has emerged to explore these ontologies as a way to compare entities annotated with the ontology concepts.
  
  \hskip20pt Semantic similarity has been extensively studied in the last decade, but some problems still persist. While there are algorithms to compare entities annotated with concepts from the same ontology, the possible ways to use \emph{more than one ontology} are still in an early phase of study. For example, comparing a metabolic pathway using both the associated molecular functions and the metabolites converted in the pathway should, in principle, yield a higher precision than would be achieved with methodologies that rely on either one of the two domains independently. Comparing concepts from \emph{different domains} and entities annotated with concepts from different domains is yet an unexplored area, but necessary to tackle multidisciplinary biomedical resources, \emph{e.g.}~to compare two clinical cases, the relationships between symptoms, diseases, blood screening results, \emph{etc.}\ should provide a more insightful and precise value of similarity.
  
  \hskip20pt In this document, I explain the basic concepts needed to understand the problem of semantic similarity, how it is being solved, and how I propose to extend this notion so that it can be applied to more than one ontology and, more significantly, to more than one domain of knowledge.
\end{itemize}


\entry
    [Sep. 2008 -- June 2010]
    [Universidade de Lisboa]
    {M.Sc. in Biochemistry}
\begin{itemize}
    \item \textbf{Thesis}: Structural and semantic similarity metrics for chemical compound similarity
    \item \textbf{Supervisor}: Prof. Dr Francisco M. Couto
    \item \textbf{Grade}: 19 out of 20
    \item \textbf{Abstract}: Over the last few decades, there has been an increasing number of attempts at creating systems capable of comparing and classifying chemical compounds based on their structure and\slash or physicochemical properties. While the rate of success of these approaches has been increasing, particularly with the introduction of new and ever more sophisticated methods of machine learning, there is still room for improvement. One of the problems of these methods is that they fail to consider that similar molecules may have different roles in nature, or, to a lesser extend, that disparate molecules may have similar roles.
    
    \hskip20pt This thesis proposes the exploitation of the semantic properties of chemical compounds, as described in the CHEBI ontology, to create an efficient system able to automatically deal with the binary classification of chemical compounds. To that effect, I developed Chym (Chemical Hybrid Metric) as a tool that integrates structural and semantic information in a unique hybrid metric. {\looseness=-1\par}
    
    \hskip20pt The work here presented shows substantial evidence supporting the effectiveness of Chym since it has outperformed all the models with which it was compared. Particularly, it achieved accuracy values of 90.9\%, 87.7\% and~84.2\% when solving three classification problems which, previously, had only been solved with accuracy values of 81.5\%, 80.6\% and~82.8\% respectively. Other results show that the tool is appropriate to use even if the problem at hand is not well represented in the CHEBI ontology. Thus, Chym shows that considering the semantic properties of a compound helps solving classification problems. {\looseness=-1\par}
    
    \hskip20pt Therefore, Chym can be used in projects that require the classification and\slash or the comparison of chemical compounds, such as the study of the evolution of metabolic pathways, drug discovery or in preliminary toxicity analysis. {\looseness=-1\par}
\end{itemize}


\entry
    [Sep. 2005 -- June 2008]
    [Universidade de Lisboa]
    {B.Sc. in Biochemistry}
\begin{itemize}
    \item \textbf{Discrimination of some grades}:
    \begin{itemize}
        \item Análise e Tratamento de Dados em Bioquímica \textit{[Data Analysis and Processing in Biochemistry]}: 19
        \item Bioquímica Computacional \textit{[Computational Biochemistry]}: 19
        \item Simulação Computacional \textit{[Computacional Simulation]}: 20
    \end{itemize}
    \item \textbf{Final grade}: 18 out of 20
\end{itemize}
